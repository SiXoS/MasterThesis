% CREATED BY DAVID FRISK, 2015

% BASIC SETTINGS
\usepackage{moreverb}								% List settings
\usepackage{textcomp}								% Fonts, symbols etc.
\usepackage{lmodern}								% Latin modern font
\usepackage{helvet}									% Enables font switching
\usepackage[T1]{fontenc}							% Output settings
\usepackage[english]{babel}							% Language settings
\usepackage[utf8]{inputenc}							% Input settings
\usepackage{amsmath}								% Mathematical expressions (American mathematical society)
\usepackage{amssymb}								% Mathematical symbols (American mathematical society)
\usepackage{graphicx}								% Figures
\usepackage{subfig}									% Enables subfigures
\numberwithin{equation}{chapter}					% Numbering order for equations
\numberwithin{figure}{chapter}						% Numbering order for figures
\numberwithin{table}{chapter}						% Numbering order for tables
\usepackage{listings}								% Enables source code listings
\usepackage{chemfig}								% Chemical structures
\usepackage[top=3cm, bottom=3cm,
			inner=3cm, outer=3cm]{geometry}			% Page margin lengths	
\usepackage{scrextend}                              % for \begin{addmargin}...
\usepackage{pdflscape}
\usepackage{wrapfig}
\usepackage{rotating}
\usepackage{epstopdf}
\usepackage{wrapfig}
\usepackage{eso-pic}								% Create cover page background
\newcommand{\backgroundpic}[3]{
	\put(#1,#2){
	\parbox[b][\paperheight]{\paperwidth}{
	\centering
	\includegraphics[width=\paperwidth,height=\paperheight,keepaspectratio]{#3}}}}
\usepackage{float} 									% Enables object position enforcement using [H]
\usepackage{parskip}								% Enables vertical spaces correctly 
%\usepackage{caption}

\usepackage{amsthm}



\newtheorem{theorem}{Theorem}[section]
\newtheorem{corollary}{Corollary}[theorem]
\newtheorem{lemma}[theorem]{Lemma}
\renewcommand\qedsymbol{$\blacksquare$}


% OPTIONAL SETTINGS (DELETE OR COMMENT TO SUPRESS)

% Disable automatic indentation (equal to using \noindent)
%\setlength{\parindent}{0cm}       

% Code lisings settings

\definecolor{dkgreen}{rgb}{0,0.6,0}
\definecolor{gray}{rgb}{0.5,0.5,0.5}
\definecolor{mauve}{rgb}{0.58,0,0.82}

\lstset{ %
  basicstyle={\small\ttfamily},        % the size of the fonts that are used for the code
  breaklines=true,                 % sets automatic line breaking
  commentstyle=\color{dkgreen},    % comment style
  frame=tb
  keepspaces=true,                 % keeps spaces in text, useful for keeping indentation of code (possibly needs columns=flexible)
  breakatwhitespace=true,              
  keywordstyle=\color{blue},       % keyword style
  language=Java,                 % the language of the code
  otherkeywords={function,each,...},           % if you want to add more keywords to the set
  numbers=left,                    % where to put the line-numbers; possible values are (none, left, right)
  numbersep=5pt,                   % how far the line-numbers are from the code
  numberstyle=\tiny\color{gray}, % the style that is used for the line-numbers
  rulecolor=\color{black},         % if not set, the frame-color may be changed on line-breaks within not-black text (e.g. comments (green here))
  showtabs=false,                  % show tabs within strings adding particular underscores
  stepnumber=1,                    % the step between two line-numbers. If it's 1, each line will be numbered
  stringstyle=\color{mauve},     % string literal style
  tabsize=2,	                   % sets default tabsize to 2 spaces
  title=\lstname                   % show the filename of files included with \lstinputlisting; also try caption instead of title
}


% Caption settings (aligned left with bold name)
\usepackage[labelfont=bf, textfont=normal,
			justification=justified,
			singlelinecheck=false]{caption} 		

		  	
% Activate clickable links in table of contents  	
\usepackage{hyperref}								
\hypersetup{colorlinks, citecolor=black,
   		 	filecolor=black, linkcolor=black,
    		urlcolor=black}


% Define the number of section levels to be included in the t.o.c. and numbered	(3 is default)	
\setcounter{tocdepth}{2}							
\setcounter{secnumdepth}{3}	


% Chapter title settings
\usepackage{titlesec}		
\titleformat{\chapter}[display]
  {\Huge\bfseries\filcenter}
  {{\fontsize{50pt}{1em}\vspace{-4.2ex}\selectfont \textnormal{\thechapter}}}{1ex}{}[]


% Header and footer settings (Select TWOSIDE or ONESIDE layout below)
\usepackage{fancyhdr}								
\pagestyle{fancy}  
\renewcommand{\chaptermark}[1]{\markboth{\thechapter.\space#1}{}} 


% Select one-sided (1) or two-sided (2) page numbering
\def\layout{2}	% Choose 1 for one-sided or 2 for two-sided layout
% Conditional expression based on the layout choice
\ifnum\layout=2	% Two-sided
    \fancyhf{}			 						
	\fancyhead[LE,RO]{\nouppercase{ \leftmark}}
	\fancyfoot[LE,RO]{\thepage}
	\fancypagestyle{plain}{			% Redefine the plain page style
	\fancyhf{}
	\renewcommand{\headrulewidth}{0pt} 		
	\fancyfoot[LE,RO]{\thepage}}	
\else			% One-sided  	
  	\fancyhf{}					
	\fancyhead[C]{\nouppercase{ \leftmark}}
	\fancyfoot[C]{\thepage}
\fi


% Enable To-do notes
\usepackage[textsize=tiny]{todonotes}   % Include the option "disable" to hide all notes
\setlength{\marginparwidth}{2.5cm} 


% Supress warning from Texmaker about headheight
\setlength{\headheight}{15pt}		

\newcommand{\maintitle}{DANF: Approximate Neighborhood Function on Large Dynamic Graphs}
\newcommand{\subtitle}{Continuously finding changes in node centrality}

