\chapter{Methods}

The approximate neighborhood function has been studied by P. Boldi et al. in the webgraph framework \cite{webgraph}. As this framework only supports static graphs, we made extensions to support dynamic ones and an evolving approximate neighborhood function. Our research was mainly focused on the compression techniques of webgraph graphs \cite{webgraph-compression}, the HyperANF algorithm \cite{hyperanf} and the HyperANF implementation HyperBall \cite{hyperball}, including HyperLogLog counters \cite{hyperloglog}. We also studied general dynamic algorithms.

\section{Merging graphs}
The existing compression methods in webgraph are very sensitive to change. Nodes can copy neighbors from other nodes so that if the neighbors of a node change, all nodes that refer to the modified node have to be found and updated as well. Moreover, the graph is either stored in a file or in a byte buffer. None of these methods support insertion of data at an arbitrary position without expensive reallocation and repositioning of the data succeeding that position. As dynamic graphs are required, we investigated optimal methods of merging two graphs.

\section{Extending HyperANF to support dynamic graphs}
The proposed algorithm is divided into two phases. In the first phase, the proposed algorithm runs the HyperANF on the initial state of the graph. The second phase is the dynamic part, which starts immediately after the HyperANF is completed. In the second phase, the algorithm takes the information produced by the HyperANF and modifies it when new nodes and edges are added or deleted. To be able to share computations, the algorithm is optimized to handle several edge insertions in a bulk.

\section{Extending HyperLogLog to support deletions}
Support for deleting nodes and edges is desired, but requires a method to remove elements from the HyperLogLog counters. We studied methods to delete elements from HyperLogLog counters. However, we observed that a satisfactory result required a more careful analysis, which was deemed out of scope. This seems like a drastic choice to make, but many data-streams are of the type "append-only", in which the resulting graph only allows node and edge additions.

An alternative way of supporting deletions by making some recalculations was investigated. Due to limited time it has not been implemented. The method is mentioned as a suggestion for future study in section \ref{sec:future_work}.

\section{Benchmarks} 
Several benchmarks on different parts of the algorithm have been performed to ensure that the methods most suitable to our needs are used. The benchmarks included comparing both time and memory usage. On several occasions, a trade-off between memory and computation speed had to be chosen. Speed is the first priority in this study, although the storage has to be kept reasonable. 

The proposed algorithm was compared to HyperANF. The comparison was done by inserting different amounts of edges into DANF, while recalculating HyperANF. Different sized graphs are tested to analyze how the comparison scales. For better accessibility, the data-sets used to perform the benchmarks will be publicly available.

